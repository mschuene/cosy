% Created 2014-06-08 Sun 23:41
\NeedsTeXFormat{LaTeX2e}
\documentclass[11pt,a4paper,oneside]{scrartcl}
\usepackage[ngerman, germanb]{babel}
\usepackage[utf8]{inputenc}
\usepackage{listings}
\usepackage{DejaVuSansMono}
\usepackage{color}
\usepackage{fancyhdr}
\usepackage{amsmath}
\usepackage{amssymb}
\usepackage[colorlinks=true,linkcolor=black]{hyperref}
\usepackage{graphicx}
\hypersetup{
    colorlinks=true,        % false: boxed links; true: colored links
    urlcolor=blue           % color of external links
}
\usepackage[margin=2.5cm]{geometry} %ist für die Größere der Seite da, wie groß darf der Textkörper sein.

% mathe makros
\newcommand{\R}{\mathbb{R}}
\newcommand{\N}{\mathbb{N}}
\newcommand{\Z}{\mathbb{Z}}
\newcommand{\Q}{\mathbb{Q}}
\newcommand{\C}{\mathbb{C}}
\newcommand{\qed}{$ \hfill \Box $}
\newcommand{\im}[1]{\text{Im}\left(#1\right)}
\newcommand{\re}[1]{\text{Re}\left(#1\right)}

%\renewcommand*\familydefault{\sfdefault}
\let\oldtexttt = \texttt
\renewcommand{\texttt}[1]{\oldtexttt{\footnotesize #1}}

% enlarge page
\geometry{tmargin=25mm,bmargin=25mm,lmargin=23mm,rmargin=23mm}

% skip between paragraphs
\setlength{\parskip}{1ex}
% ... and no indentation at start of a new paragraph
\setlength{\parindent}{0ex}

%Layout for lstlistings
\definecolor{gray}{rgb}{0.6,0.6,0.6}
\lstloadlanguages{java} % Java sprache laden, notwendig wegen option 'savemem'
\lstset{ %
	language=java,	
	basicstyle=\ttfamily\scriptsize,
	numbers=left,
	numberstyle=\tiny\ttfamily,
	numbersep=6pt,
	literate=%
		{Ö}{{\"O}}1
		{Ä}{{\"A}}1
		{Ü}{{\"U}}1
		{ß}{{\ss}}2
		{ü}{{\"u}}1
		{ä}{{\"a}}1
		{ö}{{\"o}}1,
	showspaces=false,
	showtabs=false,
	showstringspaces=false,
	keywordstyle=\bfseries,
	tabsize=4,
	frame=l,
	aboveskip=12pt,
	extendedchars=true,
	stringstyle=\ttfamily,
	commentstyle=\itshape\color{gray},
	postbreak=\space,
	breakindent=5pt,
	breaklines,
	xleftmargin=20pt}
	
% überschriften und son zeug
\pagestyle{fancy}
\lhead{} \chead{} \rhead{} 
\lfoot{} \cfoot{\thepage} \rfoot{} 
\renewcommand{\headrulewidth}{0.4pt} 
\usepackage[utf8]{inputenc}
\usepackage[T1]{fontenc}
\usepackage{fixltx2e}
\usepackage{graphicx}
\usepackage{longtable}
\usepackage{float}
\usepackage{wrapfig}
\usepackage{rotating}
\usepackage[normalem]{ulem}
\usepackage{amsmath}
\usepackage{textcomp}
\usepackage{marvosym}
\usepackage{wasysym}
\usepackage{amssymb}
\usepackage{hyperref}
\tolerance=1000
\author{Maik Schünemann}
\date{\today}
\title{Cognitive Systems Excercise 3}
\hypersetup{
 pdfkeywords={},
  pdfsubject={},
  pdfcreator={Emacs 24.3.1 (Org mode 8.2.6)}}
\begin{document}

\maketitle
\tableofcontents


\rule{\linewidth}{0.5pt}
\section{Excercise 1}
\label{sec-1}

\subsection{Concept of the extension}
\label{sec-1-1}
\subsubsection{Additional Components}
\label{sec-1-1-1}
We added group recognizer components to our cognitive
architecture, one for each group of proximity.
They extend the counting/recognizing visual routines - 
functionality of excercise 2. 
\subsubsection{Interplay between the components}
\label{sec-1-1-2}
To model human behaviour, we only concentrate on the filled
regions of the visual stimulus. We extended the peripheral view
component to quickly scan the picture to find the regions of 
the stimuli that are filled. 
On this regions we invoke the actual algorithm to count the 
groups (implemented in the components above).
If we have found a non-empty cell our system looks at all adjacent
cells if they also belong to the group. It does this until it has
found the whole group the cell belongs to.
It then continues looking for other groups but knows what cells it
has looked at before and doesn't revisit them.
The components for recognizing groups of proximity, color, shape 
only behave differently when they decide whether the adjacent cell
belongs to the same group or not

\subsubsection{Examples: See the attached Screenshots}
\label{sec-1-1-3}

\section{Excercise2}
\label{sec-2}
We introduced the special object 'obscured' to our cognitive
architecture. A obscured object behaves like a undetermined 
cell in the array. This lets us handle obscured objects uniformely
in the whole cognitive architecture:
\begin{itemize}
\item The regions detected from the peripheral view are increased to 
include the obscured objects because they could obscure a cell
of interest
\item When looking at cells for visual routines we allow obscured cells
as long as we have more known cells than obscured cells.
When we allow obscured cells in that way we have determined some
aspects of them, for example when we recognized a 2x2 square of 
red circles with one obscured object in them we have determined 
for our system that this obscured object is a red circle. This 
means that our system never counts a obscured object as two 
different things
\item because of the uniform handling of obscured objects they are also
supported when counting groups.
\item to count the number of obscured objects we let the peripheral view
filter for regions that contain obscured objects and include all 
directly reachable cells from the obscured objects in them. 
On this region we count all objects/recognize the visual routines
and count how much had obscured objects in them
\end{itemize}


\section{Examples}
\label{sec-3}
% Emacs 24.3.1 (Org mode 8.2.6)
\end{document}