% Created 2014-06-04 Wed 23:09
\NeedsTeXFormat{LaTeX2e}
\documentclass[11pt,a4paper,oneside]{scrartcl}
\usepackage[ngerman, germanb]{babel}
\usepackage[utf8]{inputenc}
\usepackage{listings}
\usepackage{DejaVuSansMono}
\usepackage{color}
\usepackage{fancyhdr}
\usepackage{amsmath}
\usepackage{amssymb}
\usepackage[colorlinks=true,linkcolor=black]{hyperref}
\usepackage{graphicx}
\hypersetup{
    colorlinks=true,        % false: boxed links; true: colored links
    urlcolor=blue           % color of external links
}
\usepackage[margin=2.5cm]{geometry} %ist für die Größere der Seite da, wie groß darf der Textkörper sein.

% mathe makros
\newcommand{\R}{\mathbb{R}}
\newcommand{\N}{\mathbb{N}}
\newcommand{\Z}{\mathbb{Z}}
\newcommand{\Q}{\mathbb{Q}}
\newcommand{\C}{\mathbb{C}}
\newcommand{\qed}{$ \hfill \Box $}
\newcommand{\im}[1]{\text{Im}\left(#1\right)}
\newcommand{\re}[1]{\text{Re}\left(#1\right)}

%\renewcommand*\familydefault{\sfdefault}
\let\oldtexttt = \texttt
\renewcommand{\texttt}[1]{\oldtexttt{\footnotesize #1}}

% enlarge page
\geometry{tmargin=25mm,bmargin=25mm,lmargin=23mm,rmargin=23mm}

% skip between paragraphs
\setlength{\parskip}{1ex}
% ... and no indentation at start of a new paragraph
\setlength{\parindent}{0ex}

%Layout for lstlistings
\definecolor{gray}{rgb}{0.6,0.6,0.6}
\lstloadlanguages{java} % Java sprache laden, notwendig wegen option 'savemem'
\lstset{ %
	language=java,	
	basicstyle=\ttfamily\scriptsize,
	numbers=left,
	numberstyle=\tiny\ttfamily,
	numbersep=6pt,
	literate=%
		{Ö}{{\"O}}1
		{Ä}{{\"A}}1
		{Ü}{{\"U}}1
		{ß}{{\ss}}2
		{ü}{{\"u}}1
		{ä}{{\"a}}1
		{ö}{{\"o}}1,
	showspaces=false,
	showtabs=false,
	showstringspaces=false,
	keywordstyle=\bfseries,
	tabsize=4,
	frame=l,
	aboveskip=12pt,
	extendedchars=true,
	stringstyle=\ttfamily,
	commentstyle=\itshape\color{gray},
	postbreak=\space,
	breakindent=5pt,
	breaklines,
	xleftmargin=20pt}
	
% überschriften und son zeug
\pagestyle{fancy}
\lhead{} \chead{} \rhead{} 
\lfoot{} \cfoot{\thepage} \rfoot{} 
\renewcommand{\headrulewidth}{0.4pt} 
\usepackage[utf8]{inputenc}
\usepackage[T1]{fontenc}
\usepackage{fixltx2e}
\usepackage{graphicx}
\usepackage{longtable}
\usepackage{float}
\usepackage{wrapfig}
\usepackage{rotating}
\usepackage[normalem]{ulem}
\usepackage{amsmath}
\usepackage{textcomp}
\usepackage{marvosym}
\usepackage{wasysym}
\usepackage{amssymb}
\usepackage{hyperref}
\tolerance=1000
\author{Maik Schünemann}
\date{\today}
\title{Cognitive Systems model}
\hypersetup{
 pdfkeywords={},
  pdfsubject={},
  pdfcreator={Emacs 24.3.1 (Org mode 8.2.6)}}
\begin{document}

\maketitle
\tableofcontents


\rule{\linewidth}{0.5pt}

\section{Introduction}
\label{sec-1}
this program evolved from a series of excercises from the course
Cognitive Systems. It models how human perceive and group objects
in a picture.

It demonstrates the concepts only \ldots{} the tasks it has to solve are
counting objects in a picture which match certain criteria in their
shapes and color.
\section{Internal representations}
\label{sec-2}
\subsection{Picture}
\label{sec-2-1}
A picture is stored as a 2-dimensional array of cells. Each cell is
just a a map with the keys :color and :shape and the values are
just keywords. A example picture can be represented like this:
\begin{verbatim}
(ns cosy.input
  (:require [quil.core :refer :all]
	    [clojure.test.check :as tc]
	    [clojure.test.check.generators :as gen]
	    [clojure.test.check.properties :as prop]))

(defn setup []
  (smooth)                         
  (frame-rate 24)                  
  (background 200))                

(def example-array
  [[{} {} {} {} [:black :circle] {} {} {}]
   [{} {} {} {} {} {} [:blue :circle] {}]
   [[:yellow :square] [:blue :square] {} {} {} {} {} {}]
   [{} [:red :circle] [:blue :circle] {} {} {} {} {}]
   [{} [:black :circls] {} {} [:black :circle] {} {} [:blue :circle]]
   [[:blue :circle]{}{}[:red :triangle]{}{}[:red :circle][:blue :quare]]
   [{} {} [:blue :circle] {} [:yellow :square] {} {} {}]
   [[:red :circle] {} {} {} {} {} {} {}]])

(alter-var-root #'*out* (constantly *out*))

(def cellwidth 50)
(def cellheight 50)

(def rgb-color {:red [255 0 0]
		:blue [0 0 255]
		:green [0 255 0]
		:yellow [255 255 0]
		:black [0 0 0]})

(defn get-rgb-color [cell]
  (get rgb-color (first cell) [255 255 255]))

(def draw-funcs
  {:square #(rect (+ 5 %1) (+ 5 %2)
		  (- cellwidth 10)
		  (- cellheight 10))
   :circle #(ellipse (+ 25 %1) (+ 25 %2)
		     (- cellwidth 10)
		     (- cellheight 10))
   :triangle #(triangle (+ %1 25) %2
			(+ %1 5) (+ %2 40)
			(+ %1 45) (+ %2 40)) })

(defn draw-shape [shape x y]
  ((get draw-funcs shape (constantly nil)) x y))

(defn visualize-array [array]
  ;;each cell will be 25x25 pixel big
  (doseq[[i y] (map-indexed
		vector (range 0 (* cellwidth (count array))
			      cellwidth))
	 [j x] (map-indexed
		vector (range 0 (* cellheight (count (first array)))
			      cellheight))]
    (fill (apply color (get-rgb-color (get-in array [i j]))))
    (draw-shape (second (get-in array [i j])) x y)))

(defn draw []
  (rect 0 0 25 25)
  (rect 25 0 25 25))

(defsketch example                 
  :title "cosy again" 
  :setup setup                     
  :draw (fn [] (visualize-array example-array))
  :size [400 400])              



(defn gen-array []
  (let [x (gen/choose 5 10)
	y (gen/choose 5 10)
	s (gen/elements [:circle :square :triangle])
	c (gen/elements [:black :blue :red :yellow :green])]
    (as-> (gen/tuple x y) x)))
\end{verbatim}


\section{Components}
\label{sec-3}
The system needs the following components to model the human behaviour
\begin{itemize}
\item perception - from a picture to informations contained in it
\item Memory - The human equivalent of storage/cache
\item Processing unit - the human cpu
\end{itemize}

\subsection{Perception}
\label{sec-3-1}
The perception is the most important cognitive component when the
goal is to model the human behaviour.
Scanning the shown picture top to botton left to right is trivial
to implement and efficient for a computer, but a human does not
perceive the world in this way.
A Human automatically filters much information that is not needed
for the task at hand and therefore does \textbf{not} look at each cell in
order to count objects.

\subsubsection{How does a Human perceive and process a visual input to count objects in it}
\label{sec-3-1-1}
Take this picture for example:
\href{cosy.png}{A sample visual stimuli}
When looking at the picture we instantly see where the colors are,
so when the task is to count the red circles we are not even
concerned about the top right area of the picture. Shapes are a
little harder to recognize.
How do we do this? 
\begin{itemize}
\item We cannot focus on the whole picture all at once but only see
sharply at the center of our viewpoint. In the rest of our
view we can't see details but we can recognize where what colors
are on the picture.
\item afterwards, we know where to focus and then move our eyes so
that we can recognize the details in the areas.
\item Even the focus-view does not look sequentially at each cell of
the picture but can scan a whole (presumably circle but
simplified to a rectangular area for this project) area at once
\item We perform thus the minimal eye movement in order to focus at
each interesting area one time and are so filtering a large part
of the picture
\end{itemize}

\subsubsection{Implementation}
\label{sec-3-1-2}
\begin{enumerate}
\item Simplifications/Assumptions
\label{sec-3-1-2-1}
We are going to assume that our system only has to process images
that aren't too big so that the whole picture fits in the
peripheral view field and we can quickly perceive where certain
colors are.
With focus view, we are able to process a small square area of
the image at once (say 4x4 cells)
\item Peripheral view - recognize boundaries for regions of color
\label{sec-3-1-2-2}
\end{enumerate}
% Emacs 24.3.1 (Org mode 8.2.6)
\end{document}