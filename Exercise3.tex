% Created 2014-06-04 Wed 23:09
\NeedsTeXFormat{LaTeX2e}
\documentclass[11pt,a4paper,oneside]{scrartcl}
\usepackage[ngerman, germanb]{babel}
\usepackage[utf8]{inputenc}
\usepackage{listings}
\usepackage{DejaVuSansMono}
\usepackage{color}
\usepackage{fancyhdr}
\usepackage{amsmath}
\usepackage{amssymb}
\usepackage[colorlinks=true,linkcolor=black]{hyperref}
\usepackage{graphicx}
\hypersetup{
    colorlinks=true,        % false: boxed links; true: colored links
    urlcolor=blue           % color of external links
}
\usepackage[margin=2.5cm]{geometry} %ist für die Größere der Seite da, wie groß darf der Textkörper sein.

% mathe makros
\newcommand{\R}{\mathbb{R}}
\newcommand{\N}{\mathbb{N}}
\newcommand{\Z}{\mathbb{Z}}
\newcommand{\Q}{\mathbb{Q}}
\newcommand{\C}{\mathbb{C}}
\newcommand{\qed}{$ \hfill \Box $}
\newcommand{\im}[1]{\text{Im}\left(#1\right)}
\newcommand{\re}[1]{\text{Re}\left(#1\right)}

%\renewcommand*\familydefault{\sfdefault}
\let\oldtexttt = \texttt
\renewcommand{\texttt}[1]{\oldtexttt{\footnotesize #1}}

% enlarge page
\geometry{tmargin=25mm,bmargin=25mm,lmargin=23mm,rmargin=23mm}

% skip between paragraphs
\setlength{\parskip}{1ex}
% ... and no indentation at start of a new paragraph
\setlength{\parindent}{0ex}

%Layout for lstlistings
\definecolor{gray}{rgb}{0.6,0.6,0.6}
\lstloadlanguages{java} % Java sprache laden, notwendig wegen option 'savemem'
\lstset{ %
	language=java,	
	basicstyle=\ttfamily\scriptsize,
	numbers=left,
	numberstyle=\tiny\ttfamily,
	numbersep=6pt,
	literate=%
		{Ö}{{\"O}}1
		{Ä}{{\"A}}1
		{Ü}{{\"U}}1
		{ß}{{\ss}}2
		{ü}{{\"u}}1
		{ä}{{\"a}}1
		{ö}{{\"o}}1,
	showspaces=false,
	showtabs=false,
	showstringspaces=false,
	keywordstyle=\bfseries,
	tabsize=4,
	frame=l,
	aboveskip=12pt,
	extendedchars=true,
	stringstyle=\ttfamily,
	commentstyle=\itshape\color{gray},
	postbreak=\space,
	breakindent=5pt,
	breaklines,
	xleftmargin=20pt}
	
% überschriften und son zeug
\pagestyle{fancy}
\lhead{} \chead{} \rhead{} 
\lfoot{} \cfoot{\thepage} \rfoot{} 
\renewcommand{\headrulewidth}{0.4pt} 
\usepackage[utf8]{inputenc}
\usepackage[T1]{fontenc}
\usepackage{fixltx2e}
\usepackage{graphicx}
\usepackage{longtable}
\usepackage{float}
\usepackage{wrapfig}
\usepackage{rotating}
\usepackage[normalem]{ulem}
\usepackage{amsmath}
\usepackage{textcomp}
\usepackage{marvosym}
\usepackage{wasysym}
\usepackage{amssymb}
\usepackage{hyperref}
\tolerance=1000
\author{Maik Schünemann}
\date{\today}
\title{Cognitive Systems model}
\hypersetup{
 pdfkeywords={},
  pdfsubject={},
  pdfcreator={Emacs 24.3.1 (Org mode 8.2.6)}}
\begin{document}

\maketitle
\tableofcontents


\rule{\linewidth}{0.5pt}

\section{Task 1}
\subsection{Concept of the extension}
First we searched for a way to count all groups of type proximity in a picture. The idea was to use the peripheral view from the last exercise to find all clusters. An object is in a cluster if it has a directly adjacent object that is in that cluster. Also clusters have more than one object in them, because they otherwise would not be a group. After that we count the clusters just like we would count single objects.\\
When the system is asked to count all groups of similarity of color or form, it first clusters the whole picture, just like when we search of type proximity. Now we look at every cluster and search for groups of type form or color. Everytime we find one, the memory will increment its number of groups found.
\section{Task 2}
\subsection{Concept of the extension}
We decided, that an object is obscured, when more than half of its objects are not obscured. For example, a 4 times 4 square would be obscured if 10 objects are not obscured, but would not be obscured if 10 objects are obscured.\\
In exercise 2 we already implemented a way to recognize certain object like squares and lines. We figured that we could use that to find obscured objects as well.  
\end{document}