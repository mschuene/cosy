% Created 2014-06-10 Tue 22:53
\documentclass[presentation]{beamer}
\usepackage[utf8]{inputenc}
\usepackage[T1]{fontenc}
\usepackage{fixltx2e}
\usepackage{graphicx}
\usepackage{longtable}
\usepackage{float}
\usepackage{wrapfig}
\usepackage{rotating}
\usepackage[normalem]{ulem}
\usepackage{amsmath}
\usepackage{textcomp}
\usepackage{marvosym}
\usepackage{wasysym}
\usepackage{amssymb}
\usepackage{hyperref}
\tolerance=1000
\usetheme{default}
\author{Maik Schünemann}
\date{\today}
\title{Excercise 1 presentation}
\begin{document}

\maketitle
\begin{frame}{Outline}
\tableofcontents
\end{frame}

\begin{frame}[label=sec-1]{Outline}
\begin{itemize}
\item Components and Processes
\item Dealing with the example problems
\item Examples
\end{itemize}
\end{frame}
\begin{frame}[label=sec-2]{Components and Processes}
To be able to process visual stimuli we need the following components:
\begin{itemize}
\item Perceptual System
Responsibilities:
\begin{itemize}
\item Build a model of the visual stimuli suitable to further processing.
\item Build a model of the given task
\end{itemize}
\item Strategical System
Responsibilities:
\begin{itemize}
\item Models the thinking process
\item Designs a strategy of how to approach the problem
\end{itemize}
\item Processing System
Responsibilities:
\begin{itemize}
\item Executes the chosen strategy
\item Stores results in the memory system
\end{itemize}
\end{itemize}
\end{frame}

\begin{frame}[label=sec-3]{Components and Processes}
\begin{itemize}
\item Memory System
Responsibilities:
\begin{itemize}
\item In this problem domain we only need short term memory
\item Simply provides a place to store information
\end{itemize}
\end{itemize}
\begin{itemize}
\item Temporal System
\begin{itemize}
\item Gives the architecture a sense of time
\item Can cancel current actions when there is no time left
\end{itemize}
\end{itemize}
\end{frame}



\begin{frame}[label=sec-4]{Dealing with the example problems}
\begin{block}{Task is described}
\begin{itemize}
\item Perceptual System responsible for modeling and storing the model in memory
\item Strategical System decides about the strategy
\item Temporal system sets timer
\end{itemize}
\end{block}
\begin{block}{Picture is shown}
\begin{itemize}
\item Perceptual system models the system and stores model in memory
\item Processing system is invoked to execute the chosen strategy
\begin{itemize}
\item Stores intermediate results in memory
\end{itemize}
\item When no time is left gets interrupted by the memory system
\item Has chance to make a guess about the final result
\end{itemize}
\end{block}
\end{frame}

\begin{frame}[label=sec-5]{Models}
\begin{block}{Task}
\begin{itemize}
\item modeled after features it has to recognize when looking at a single cell
\end{itemize}
\end{block}

\begin{block}{Picture}
\begin{itemize}
\item Matrix of cells containing features of objects at the position
\end{itemize}
\end{block}

\begin{block}{Strategy}
determines:
\begin{itemize}
\item which order to process the cells of the picture
\item guess when no time is left
\end{itemize}
\end{block}
\begin{block}{Simple strategy}
\begin{itemize}
\item scan top to bottom left to right
\item return number of matching objects counted if no time is left
\end{itemize}
\end{block}

\begin{block}{Advanced strategy}
\begin{itemize}
\item scan top to bottom left to right
\item interpolate guess from amount seen so far on interrupt
\end{itemize}
\end{block}
\end{frame}

\begin{frame}[label=sec-6]{Examples}
Live\ldots{}..
\end{frame}
% Emacs 24.3.1 (Org mode 8.2.6)
\end{document}