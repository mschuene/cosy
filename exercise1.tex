% Created 2014-05-14 Wed 08:27
\documentclass[presentation]{beamer}
\usepackage[utf8]{inputenc}
\usepackage[T1]{fontenc}
\usepackage{fixltx2e}
\usepackage{graphicx}
\usepackage{longtable}
\usepackage{float}
\usepackage{wrapfig}
\usepackage{rotating}
\usepackage[normalem]{ulem}
\usepackage{amsmath}
\usepackage{textcomp}
\usepackage{marvosym}
\usepackage{wasysym}
\usepackage{amssymb}
\usepackage{hyperref}
\tolerance=1000
\usetheme{default}
\author{Maik Schünemann}
\date{\today}
\title{Exercise 1}
\begin{document}

\maketitle
\begin{frame}{Outline}
\tableofcontents
\end{frame}


\rule{\linewidth}{0.5pt}
\begin{frame}[label=sec-1]{Design and Implementation of a simple cognitive architecture}
For the given problem of processing visual stimuli 
we designed the following cognitive architecture:
\begin{block}{Components and Processes}
To be able to process visual stimuli we need the following components:
\begin{itemize}
\item Perceptual System
Responsibilities:
\begin{itemize}
\item Build a model of the visual stimuli suitable to further processing.
\item Build a model of the given task
\end{itemize}
\item Strategical System
Responsibilities:
\begin{itemize}
\item Models the thinking process
\item Designs a strategy of how to approach the problem
\end{itemize}
\item Processing System
Responsibilities:
\begin{itemize}
\item Executes the chosen strategy
\item Stores results in the memory system
\end{itemize}
\item Memory System
Responsibilities:
\begin{itemize}
\item In this problem domain we only need short term memory
\item Simply provides a place to store information
\end{itemize}
\item Temporal System
\begin{itemize}
\item Gives the architecture a sense of time
\item Can cancel current actions when there is no time left
\end{itemize}
\end{itemize}
\end{block}
\begin{block}{Dealing with the example problems}
Our system can solve the exercises in the following way:
When the task is described to our system, the Perceptual is
responsible for modeling it and storing the model in memory.
It also invokes the Strategical System to plan how the task is
going to be solved and stores the plan in the Memory system.
And gives the Temporal System the information of how much time it
has to perform the task.
Afterwards, when the time is set and our system is shown the
picture for the task, the Perceptual System first builds an
internal model of the picture and invokes the Processing System to
execute the chosen plan. The Processing System interacts with the
Memory system to store intermediate results.
The Temporal System is responsible for constraining the time frame
available for the Processing System and cancels it when there is no
time left. According to the chosen Strategy the (intermediate)
Result is transformed to a final answer of the system.

\begin{block}{Model of the Task}
Because we only have to deal with counting tasks which check for
certain features of the elements in a picture, we can model the
task by the features it has to recognize when looking at a cell of
the picture.
\end{block}

\begin{block}{Model of the Picture}
The picture can be modeled by a matrix of cells containing the
features of the object at the given position.
\end{block}

\begin{block}{Model of the Strategy}
The Strategy determines in which order the picture is processed
and what to do when no time if left.
Example Strategies are:
\begin{itemize}
\item look at cells from top to bottom left to right and count how
often the task accepted a cell. If no time is left simply report
the current information
\item like above but interpolate from the amount of the picture
processed so far a guess for the final answer
\end{itemize}
\end{block}
\end{block}
\end{frame}


\begin{frame}[label=sec-2]{Examples:}
We will demonstrate the examples live during the presentation of the assignment.
For the examples, we chosed small arrays as inputs for our system.
We simulate the behaviour of running out of time by adding timeouts to the 
executer
\begin{block}{1}
Picture:
[[\#\{:blue :square\} \#\{:red :hexagon\} \#\{:yellow :circle\} \#\{:yellow :triangle\}]
 [\#\{\} \#\{:blue :square\} \#\{:blue :square\}\#\{:blue :square]])


Search for: blue square


Answer: 4


If the available time is enough the system will, of course, give the right answer.
With the simple strategy, the system will return the number of matching items 
seen so far, so if it had only time to look at the first item, it will return
1
\end{block}
\begin{block}{2}
Picture
[[\#\{:blue :square\} \#\{:red :square\} \#\{:yellow :square\} \#\{:yellow :square\}]
 [\#\{\} \#\{:blue :square\} \#\{:blue :square\} \#\{:blue :square\}]])


Search for: blue square


Answer:4


The system can distinguish similar shapes. With enough time it will also give the right answer.
Same goes for similar colors.
\end{block}

\begin{block}{3}
Picture:
[[\#\{:blue :square\} \#\{:red :hexagon\} \#\{:blue :square\}\#\{:yellow :triangle\}] 
[[\#\{:blue :square\} \#\{:brown :triangle\} \#\{:blue :square\}\#\{:yellow :hexagon\}] 
[[\#\{:blue :square\} \#\{:orange :triangle\} \#\{:blue :square\}\#\{:green :circle\}] 
[[\#\{:blue :square\} \#\{:yellow :square\} \#\{:blue :square\}\#\{:yellow :square\}] 


Search for: blue square


If we test it with the advanced strategy and don't give it enough time, it will 
guess at timeout how much blue squares are there interpolating from how much
it has seen so far and how much of the array it has traversed.
This is a good strategy here because we have a homogenous distribution of blue 
squares here.
\end{block}
\end{frame}
% Emacs 24.3.1 (Org mode 8.2.6)
\end{document}